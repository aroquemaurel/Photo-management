\documentclass[a4paper, 11pt]{article}

\usepackage{xcolor}
\input{/home/aroquemaurel/cours/includesLaTeX/couleurs.tex}
\usepackage{lmodern}
\usepackage[utf8]{inputenc}
\usepackage[T1]{fontenc}
\usepackage[francais]{babel}
\usepackage[top=1.7cm, bottom=1.7cm, left=2.5cm, right=2.5cm]{geometry}
\usepackage{verbatim}
\usepackage{tikz} %Vectoriel
\usepackage{listings}
\usepackage{fancyhdr}
\usepackage{multido}
\usepackage{amssymb}
\usepackage{multicol}
\usepackage{float}
\usepackage[urlbordercolor={1 1 1}, linkbordercolor={1 1 1}, linkcolor=vert1, urlcolor=bleu, colorlinks=true]{hyperref}

\newcommand{\titre}{Gestion de photos à partir de tags}
\newcommand{\numero}{}
\newcommand{\typeDoc}{Projet}
\newcommand{\module}{Système 1}
\newcommand{\sigle}{systeme}
\newcommand{\semestre}{3}

\input{/home/satenske/cours/includesLaTeX/l2/tddm.tex}
\input{/home/aroquemaurel/cours/includesLaTeX/listings.tex} %prise en charge du langage C 
\input{/home/aroquemaurel/cours/includesLaTeX/remarquesExempleAttention.tex}
\input{/home/aroquemaurel/cours/includesLaTeX/polices.tex}
\input{/home/aroquemaurel/cours/includesLaTeX/affichageChapitre.tex}
\newcommand{\exiv}{\texttt{exiv2}}
\makeatother
\begin{document}
	\maketitle
	\section{Scripts Simples}
	\subsection{\texttt{Date\_prise\_de\_vue.sh [-o] fichier\_photo [fichier\_photo \ldots]}}
	Ce script assez simple à permis de prendre en main la coommande \texttt{exiv2}, c'est également avec ce script que des scripts utiles et simples ont été crées, 
	ceux-ci sont utilisés dans plusieurs scripts du projet. Cf section \ref{scriptsUtils}.

	Pour obtenir la date de prise de vue d'une photo, il faut utiliser le tag \texttt{Exif.Photo.DateTimeOriginal}, la commande \exiv{} possède un argument qui permet
	d'afficher les informations concernant un tag, ainsi à l'aide de la commande \texttt{exiv2 photo -g Exif.Photo.DateTimeOriginal} nous possédons 
	toutes les informations de prises de vue, ensuite à l'aide d'un cut nous pouvons en retirer les informations nécessaires.

	Étant donné que nous pouvons choisir de donner plusieurs photos en paramètre, la commande ci-dessus est placé dans un \texttt{for}.
	\lstinputlisting[numbers=none, language=sh, basicstyle=\scriptsize\ttfamily, mathescape=false,
	caption=Exemple d'utilisation du script \texttt{Date\_prise\_de\_vue.sh}]{scripts/exemple1.sh}

	\subsection{\texttt{Met\_date\_dans\_nom.sh [-a] fichier\_photo}}
	Ce script nécessite le script précédent, en effet afin de mettre la date dans le nom, je récupère la date à l'aide de \texttt{Date\_prise\_de\_vue.sh}.

	Ce script peut prendre en paramètre plusieurs photos bien que ça ne soit pas spécifié dans le synopsis par la présence de \ldots, en effet, il était très 
	simple à mettre en oeuvre la présence de plusieurs paramètre (un simple \texttt{for}), et cela peut sembler très pratique.
	\lstinputlisting[numbers=none, language=sh, basicstyle=\scriptsize\ttfamily, mathescape=false,
	caption=Exemple d'utilisation du script \texttt{Met\_date\_dans\_nom.sh}]{scripts/exemple2.sh}

	\subsection{\texttt{Change\_description.sh fichier\_photo description}}
	Contrairement aux autres scripts, celui-ci ne prend qu'un seul paramètre : La description et la photo pour laquelle il faut changer la description.

	Afin de pouvoir changer la description d'une image, j'ai recherché dans le man d'exiv2 et ai trouvé la commande suivante qui faisait ce dont j'avais besoin : 

	\texttt{exiv2 -M''set Exif.Image.ImageDescription la nouvelle description'' image.jpg}
	\lstinputlisting[numbers=none, language=sh, basicstyle=\scriptsize\ttfamily, mathescape=false,
	caption=Exemple d'utilisation du script \texttt{Change\_description.sh}]{scripts/exemple3.sh}
	\subsection{\texttt{Change\_date\_modif.sh fichier\_photo}}
	Pour ce script, je me suis permis d'avoir la possibilité de passer en paramètre une liste de photos et non une unique photo, en effet, cela était très simple
	à mettre en œuvre\footnote{un simple \texttt{for}} et cela peut être extrêmement pratique de faire passer \texttt{photo/*} par exemple.

	Afin de modifier la date de modification d'un fichier, des recherches Google m'ont permis de découvrir que la commande \texttt{touch} possédait un argument
	qui permettait de modifier celle-ci : \texttt{touch -t yyyyMMJJhhmm[.ss]} modifie la date de modification  au \texttt{jj/MM/yyyy} à \texttt{hh:mm:ss}

	Ce script nécessite le script \texttt{Date\_prise\_de\_vue.sh}, en effet celui-ci est utilisé pour obtenir la date de prise de vue.

	\section{Scripts utiles}\label{scriptsUtils}
	Afin de limiter la redondance au maximum, j'ai créer trois scripts qui sont appelés dans chacun de scripts, ceux-ci sont des scripts de base de
	vérifications.
	\subsection{\texttt{haveArgument.sh}}
	Ce script vérifie que le premier paramètre est présent dans la liste suivante de paramètre, ce script est utile afin de vérifier que les options sont
	présentes, ou absente d'un appel de script.

	Il est toujours utilisé de la façon suivante : 
	\lstinputlisting[numbers=none, language=sh, basicstyle=\scriptsize\ttfamily, mathescape=false,
	caption=Exemple d'utilisation du script \texttt{haveArgument.sh}]{scripts/exemple7.sh}
	La variable \texttt{var} valant 0 si l'argument est présent et 1 sinon.
	\subsection{\texttt{isAccessibleDirectory.sh}}
	Ce script comme son nom l'indique sert à vérifier qu'un répertoire est accessible en lecture, dans le cas d'une accessibilité en écriture, je rajoute un
	test manuellement.
	\subsection{\texttt{isAccessibleFile.sh}}
	Ce script comme son nom l'indique sert à vérifier qu'un fichier est accessible en lecture, dans le cas d'une accessibilité en écriture, je rajoute un
	test manuellement.
	\section{Scripts complexes}
	\subsection{\texttt{Range\_selon\_date\_et\_description.sh [-c | -a] fichier\_photo [répertoire]}}
	Cette fonction comporte un certain nombre de sous-programme afin de simplifier sa compréhension.
	\subsubsection{\texttt{haveGoodNbArguments}}
	Ce sous-programme sert simple à vérifier que le nombre d'argument du programme est correct, ceci à l'aide du script \texttt{util/haveArgument.sh}.
	\subsubsection{\texttt{replaceAlphanumericByUnderscore}}
	Cette fonction remplace les caractères alphanumériques par des underscore (\_) à l'aide du programme \texttt{sed}.
	\subsubsection{\texttt{directoryArgumentExistAndIsWrittable}}
	Ce sous-programme vérifie que le répertoire passé en argument existe et est accessible en écriture.
	\subsubsection{\texttt{getFileAndDirectory}}
	Ce script permet d'obtenir le répertoire de départ et le nom de la photo à déplacer, ceci en fonction des options présentes ou non, en effet, en fonction de
	ces options, les numéros d'arguments peuvent changer.
	
	Si le répertoire n'est pas spécifié dans les arguments, celui-ci est \texttt{./}.
	\subsubsection{\texttt{getGoodPartOfDate}}
	Avec \texttt{getGoodPartOfDate} nous pouvons obtenir la partie de la date de prise de vue de la variable \texttt{file}\footnote{Celle-ci est obtenur après le
	lancement du script \texttt{getFileAndDirectory}} que nous voulons en fonction de l'argument : 1 nous retourne l'année, 2 nous
	retourne le mois et 3 nous retourne le jour.
	\subsubsection{\texttt{getDescription}}
	Cette fonction nous permet d'obtenir la description de \texttt{file}, la description ayant ses caractères alphanumériques remplacés par des underscore à
	l'aide du script \texttt{replaceAlphanumericByUnderscore}.
	\subsubsection{\texttt{getDirectoriesPath}}
	Cette fonction nous retourne le chemin des répertoires à créer, ainsi celui-ci sera affiché si l'option \texttt{-a} est spécifié, et sera utilisé avec un
	\texttt{mkdir -p} dans les autres cas.
	

	\lstinputlisting[numbers=none, language=sh, basicstyle=\scriptsize\ttfamily, mathescape=false,
	caption=Exemple d'utilisation du script \texttt{Date\_prise\_de\_vue.sh}]{scripts/exemple5.sh}
	\subsection{\texttt{Range\_Mes\_Photos.sh [-d répertoire] fichier\_photo [fichier\_photo \ldots]}}
	Ce script utilise simplement le script précédent \texttt{Range\_selond\_date\_et\_description.sh} dans un \texttt{for} qui parcours la liste des arguments.
	\lstinputlisting[numbers=none, language=sh, basicstyle=\scriptsize\ttfamily, mathescape=false,
	caption=Exemple d'utilisation du script \texttt{Date\_prise\_de\_vue.sh}]{scripts/exemple6.sh}
\end{document}
